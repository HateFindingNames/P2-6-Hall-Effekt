\chapter{Versuchsdurchführung}
Vor Inbetriebnahme und vor jeder neuen Messung muss sichergestellt werden, dass die Spulen und der Eisenkerne von
Restmagnetisierung befreit werden (siehe \bild{fig:schematicEntmag}). Die Spulen werden so mit einer sinusförmigen Wechselspannung beaufschlagt, dass sich
ein effektiver Spulenstrom von \SI{1,8}{A} einstellt (überwacht durch ein in Reihe geschaltetes Amperemeter). Am
variablen Trenntransformator wird nun allmählich die Spannung auf \SI{0}{V} herunter geregelt.
\par
Anschließend muss bei Feldfreiem Luftspalt mittels Teslameter die Abwesenheit einer Restmagnetisierung überprüft werden.
Ist noch Restmagnetisierung feststellbar muss der Vorgang wiederholt werden.
\par\medskip
Weiter muss vor der ersten Messung und bei jeder Änderung des Probenstroms \(I_p\) eine \textit{Fehlspannungskompensation}
durchgeführt werden. Fertigungsbedingt kann es vorkommen, dass die Abgriffe der \textsc{Hall}-Spannung nicht exakt gegenüber
liegen. Hierdurch kommt es zu einem vom Probenstrom abhängigen Spannungsabfall entlang der Abgriffe trotz fehlenden Magnetfeldes.
Um diesen Messfehler zu kompensieren muss nach jeder Änderung des Probenstroms \(I_p\) die über die Buchsen 5 und 6 abgegriffene
\textsc{Hall}-Spannung $U_H$ mit einem geeigneten Voltmeter überwacht und gegebenenfalls durch drehen des Reglers 8 kompensiert
werden (vgl. hierzu \bild{fig:messmodul}).
\par\medskip
Wenn alle Vorbereitungen abgeschlossen sind kann das Magnetfeld eingeschaltet werden. Hierzu wird der Trenntransformator
nach \bild{fig:schematicMessung} durch eine Gleichstromquelle ersetzt. Es werden nacheinander und in dieser Reihenfolge Spulenströme
von $I=\SI{1,8}{A}$, $\SI{2,4}{A}$ und $\SI{2,95}{A}$ eingestellt, die sich jeweils einstellende magnetische Flussdichte
am Teslameter abgelesen und die sich für verschiedene Probenströme einstellenden \textsc{Hall}-Spannungen $U_H$ fest gehalten.
Darüber hinaus sind zu jedem Probenstrom die Spannungsabfälle $U_p$ über dem Probenkreis aufzuzeichnen.