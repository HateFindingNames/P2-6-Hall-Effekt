\chapter{Fazit}
Bei einem Probenstrom von \SI{0}{A} ist nach \cref{eq:UH2} eine \textsc{Hall}-Spannung von \SI{0}{V} zu erwarten. Der Offset
der \textsc{Hall}-Spannung in \cref{fig:hallspannung} lässt auf eine unsaubere Fehlspannungskorrektur schließen. Da in Folgenden
Berechnungen allerdings die Steigung eingeht und diese unempfindlich gegenüber Verschiebungen entlang der \(x\)- und \(y\)-Achse
ist, wird angenommen, dass dieser Offset eine untergeordnete Rolle spielt. Interessanter ist jedoch, dass obschon eine
Fehlspannungskorrektur gemäß Versuchsanleitung mit bei einem Probenstrom von \SI{5}{mA} durchgeführt wurde, sich die
\(x\)-Achsenabschnitte der linearen Fits aller drei Messreihen bei \(\approx \SI{2,5}{mA}\) wiederfinden. Das spricht für
einen gegenüber dem angenommenen Messfehler für \(\Delta I_p = \SI{1}{mA}\) nicht zu ignorierend höheren Messfehler von
\(250\%\). Selbiges spiegelt sich auch \cref{fig:Rprobe} wieder.
\par\medskip
Es wurde während der Messung versäumt die Dimensionen und weitere Spezifikationen der Probe aufzuzeichnen. Für die Auswertung
wurden Werte für \(l\text{, }b\text{und }h\) dem Datenblatt des Herstellers \autocite{PHYWESystemeGmbHundCo.KG.} entnommen.
Da aus diesem allerdings keine Messunsicherheiten hervorgehen wurden für die drei Längenangaben jeweils \(\pm\SI{0,1}{mm}\)
angenommen.
\par
Ebenfalls wurde vom Experimentator die Temperaturabhängigkeit der Ladungsträgerbeweglichkeit und -konzentration nicht in
Betracht gezogen. Durch \cref{eq:hallKoeff} hängt hiermit direkt der \textsc{Hall}-Koeffizient und damit die Höhe der \textsc{Hall}-Spannung
zusammen. Da der Probenstrom am Innenwiderstand der Probe eine Verlustleistung erzeugt und die Probe damit erwärmt, ist
mit fortschreitenden Messungen eine Verzerrung der \textsc{Hall}-Spannung zu rechnen.
\par\medskip
Mit einem negativen Vorzeichen des \textsc{Hall}-Koeffizienten und der hierdurch ebenfalls negativen Ladungsträgerbeweglichkeit
\(\mu\) ist anzunehmen, dass es sich bei der Probe um p-dotiertes Germanium handelt. Hier bilden \textit{Löcher} als positive
pseudo-Ladungsträger den Stromfluss. Entsprechend fließen innerhalb des Germaniumkristalls freie Elektronen in entgegen gesetzter
Richtung. Diese sind es jedoch, die unter Einfluss eines äußeren magnetischen Feldes der \textsc{Lorentz}-Kraft unterliegen
und in ihrer Bahn so abgelenkt werden, dass sich die \textsc{Hall}-Spannung wie eingangs beschrieben ausbilden kann. Siehe auch
\cref{eq:UH1}: eine Umkehr der Ausrichtung der Driftgeschwindigkeit \(\vec{v_D}\) geht mit einer Vorzeichenumkehr der Hallspannung
einher.
\par\medskip
Im allgemeinen liegen die Messunsicherheiten bei \(\approx 13-14\%\) während der für die Ladungsträgerbeweglichkeit bei
nahe \(30\%\) liegt. Über Lehr- und Demonstrationszwecke hinaus ist der Aufbau - subjektiv - als ungeeignet einzuordnen.
\par\bigskip
Zuletzt ist anzumerken, dass die Versuchsanleitung zu wünschen übrig lässt. Falsche Schaltbilder, irreführende Beschreibungen
und ausgelassene wichtige Hinweise haben sowohl den initialen Aufbau, wie auch die Durchführung und Auswertung erheblich
erschwert. Die Kürze der Bearbeitungszeit außer Acht gelassen hat dies jedoch zu einer tiefergehenden Auseinandersetzung
mit der Thematik angeregt.