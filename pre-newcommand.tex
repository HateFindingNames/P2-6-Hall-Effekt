%-------------------------
 %\renewcommand{\captionlabelfont}{\sffamily} %für "Abbildung" und "Tabelle"
 %\renewcommand{\captionfont}{\sffamily\small} %für den Text der Bildunterschriften
%  \renewcommand{\captionlabelfont}{\sffamily} 
%  \renewcommand{\captionfont}{\sffamily} %\renewcommand{\normalfont}{\sffamily} %für die Überschriften
% Fettdruck der Bezeichnung Abbildung, Tabelle
%\renewcommand{\captionlabelfont}{\bfseries}
%---------------------------------------------
%------------------ Schrifttyp in der Kopf- und Fusszeilen
\setkomafont{pageheadfoot}{\footnotesize\sffamily}
%---------------------------------------------------
%Ändern der Abbildung- und Tabellenbezeichnung (Niedermair S.157)
%_____________________________________________
\addto\captionsngerman{\renewcommand\figurename{Abb.}}
\addto\captionsngerman{\renewcommand\tablename{Tab.}}
\renewcommand\listfigurename{Abbildungen}
%_______________________________
%Betrag eines Wertes
\newcommand{\abs}[1]{\lvert #1 \rvert} 
%---------------------------------------------
\newcommand{\absatz}[1]{\textbf{\textsc{#1}}} %siehe Mittelbach S. 876ff
%----------------------------
%\newcommand{\absatz}{\par \medskip}
%______________________________
%\newcommand{\anhang}[1]{Anhang \ref{#1}, Seite \pageref{#1}}
\newcommand{\anhang}[1]{Anhang \vref{#1}}
%________________________________
\newcommand{\aufgabe}{\stepcounter{plus} Aufgabe \arabic{plus}}
%______________________________________
%compactitem
\newcommand{\bci}{\begin{compactitem}}
\newcommand{\eci}{\end{compactitem}}
%______________________________________
%
\newcommand{\bi}{\begin{itemize}}
\newcommand{\ei}{\end{itemize}}
%______________________________________
%compactenumerate
\newcommand{\bce}{\begin{compactenum}}
\newcommand{\ece}{\end{compactenum}}
%_____________________________________
%begin equation
\newcommand{\be}{\begin{equation}}
\newcommand{\ee}{\end{equation}}
%_____________________________________
%begin equation ohne Formelnummer
\newcommand{\ben}{\begin{equation*}}
\newcommand{\een}{\end{equation*}}
%_____________________________________
%begin align ohne Formelnummer
\newcommand{\ban}{\begin{align*}}
\newcommand{\ean}{\end{align*}}
%_____________________________________
%begin align
\newcommand{\ba}{\begin{align} }
\newcommand{\ea}{\end{align}}
%_____________________________________
%minpage
\newcommand{\bmp}{\begin{minipage}[t]{.47\linewidth}}
\newcommand{\emp}{\end{minipage}}
%_____________________________
%  dB
\newcommand{\db}{dB}
%  dB(A)
\newcommand{\dba}{ dB(A) }
%  dB(A) für Satzende
\newcommand{\dbap}{ dB(A)}
%_________________________________
\newcommand{\bzw}{bzw.\,}
%_______________________________
\newcommand{\dif}{\mathrm{d}}
%________________________________
% neuer Zähler
\newcounter{plus}
\setcounter{plus}{0}
%______________________________
%Für das Formelverzeichnis _____________________ Formelverzeichis ____
% Befehl umbenennen in fz
\let\fz\nomenclature
% Deutsche Überschrift
\renewcommand{\nomname}{Formelzeichen}
% Punkte zw. Abkürzung und Erklärung
\setlength{\nomlabelwidth}{.20\hsize}
\renewcommand{\nomlabel}[1]{#1 \dotfill}
% Zeilenabstände verkleinern
\setlength{\nomitemsep}{-\parsep}
%_____________________________
\newcommand{\bild}[1]{Abb. \vref{#1}}
\newcommand{\sbild}[1]{siehe Abb. \vref{#1}}
\newcommand{\bilder}[2]{Abb. \vrefrange{#1}{#2}}
\newcommand{\bildseite}[1]{Abb. \vref{#1}} % erzeugt "`Abb. nn auf Seite nn
\newcommand{\tabelle}[1]{Tab. \vref{#1}}
\newcommand{\tabellenseite}[1]{Tab. \vref{#1}} % erzeugt "`Tab. nn auf Seite nn
% für \vref ist usepackage[german]{varioref} einzufügen
%_-------------------------------- Freiraum
\newcommand{\freiraum}[1]{\begin{figure}[H]\vspace{#1\textheight}\end{figure}}
%_____________________________
% Gleichung
\newcommand{\gl}[1]{Gl.\,(\ref{#1})}
\newcommand{\sgl}[1]{siehe Gl.\,(\ref{#1})}
\newcommand{\glbereich}[2]{Gl. \vrefrange[]{#1}{#2}}
%_____________________________
% Grad Celsius
\newcommand{\grad}{\,\degC}
\newcommand{\gradC}{\,\degree}
%______________________________________
% Großbuchstaben als Indizes kleiner schreiben; spezielle im Mathemodus
\newcommand{\klein}[1]{\scriptscriptstyle{#1}}% Fettdruck der Bezeichnung 
%______________________________
%% Kasten
\newcommand{\kasten}{\fbox{\rule{0.0pt}{10pt}{{ } } }}
%______________________________
%\newcommand{\kapitel}[1]{Kapitel \ref{#1}, Seite \pageref{#1}}
\newcommand{\kapitel}[1]{Kapitel \vref{#1}}
%_____________________________
%  LAeq für den äquivalenten Dauerschallpegel
\newcommand{\laeq}{ $L_{Aeq}$ }
%  LAeq für den äquivalenten Dauerschallpegel am Satzende
\newcommand{\laeqp}{ $L_{Aeq}$}
%_________________________________
%   Linie zeichnen
\newcommand{\linie}{\rule{0.5\textwidth}{0.1pt}}
%---------------------- LaTeX
\newcommand{\lt}{\LaTeX\,\,}
%----------------------------
\newcommand{\nl}{\newline}
%__________________________________
%  multicolumn für Tabellen
\newcommand{\mc}{\multicolumn}
%% \mc{1}{c}{Text}
%_________________________________
%    Parallel
\newcommand{\pl}[1]{\ParallelLText{#1}}
\newcommand{\pr}[1]{\ParallelRText{#1}}
\newcommand{\pp}{\ParallelPar}
%------------ rot unterstrichen
\newcommand{\rotunterstrichen}[1]{\textcolor{red}{\underline{\textcolor{black}{#1}}}}
%________________ Realteil
\newcommand{\real}[1]{\text{Re}\left\{#1\right\}}
%________________________________--
\newcommand{\seite}[1]{Seite \pageref{#1}}
\newcommand{\seiten}[2]{\vpagerefrange{#1}{#2}}
%----------------------------
%        TEXT Rot
\newcommand{\textrot}[1]{\textcolor{red}{#1}}
%______________________________
%Abkürzung für \multicolumn
\newcommand{\tab}[2]{\multicolumn{1}{#1}{#2}}
%_____________________________________
% doppelt unterstreichen
\newcommand{\unterstreichen}[1]{\underline{\underline{#1}}}
% einfach unterstreichen
\newcommand{\ul}[1]{\underline{#1}}
%______________________________
%kurze Verbatimausgabe
%\MakeShortVerb{\|} %mittelbach S.160 mit \usepackage{shortvrb}
%_______________________________________
% vspace
\newcommand{\vsf}{\vspace{5pt}}
%_________________________________
\newcommand{\zb}{z.B.\,}
\newcommand{\idr}{i.d.R.\,}
%_______________________________
% Zähler-einfach
\newcounter{req}
\newcommand{\zaehler}[1]{\refstepcounter{req}{#1} \thereq}
%% Beispielaufzählung \zaehler{Beispiel}\\
%_____________________________________